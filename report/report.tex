\documentclass{MathematicaReport}

% PACKAGES
\usepackage{parskip}
\usepackage{mathtools}

\place{Gliwice}
\date{\today}
\tasknumber{1}
\title{Comparative analysis of methods for approximating functions}
\author{Jakub Pietrasik, Bartosz Lewandowicz, Szymon Hankus}
\field{Informatics -- sem. V}


\begin{document}
\maketitle

\section{Project objective}
The goal of the project is to explain, present, and compare three methods of
approximating functions either from a set of given points or by approximating
the function itself. The methods that we chose are:
\begin{enumerate}
	\item Interpolation using Lagrange polynomials
	\item Fourier series
	\item Linear regression
\end{enumerate}

\section{Description} 
\subsection{Lagrange polynomials}
\textbf{Lagrange polynomials} are polynomials \( w(x) \) that interpolate a 
given set of points
\( P = \{ (x_0, y_0)\)\(,\ (x_1, y_1)\)\(,\ldots\) \(,\ (x_n, y_n) \} \)
Which means that they satisfy the following condition:

\[
	\forall_{(x_i, y_i) \in P}\ w(x_i) = y_i
\]

In other words, when a polynomial \textbf{interpolates} the dataset, it passes 
through all of the points in said dataset.
The Lagrange interpolating polynomial is the unique polynomial of lowest degree 
that interpolates a given set of data

Langrage polynomials for a given set of data are calculated using the following
formula
\[
	w(x) = \sum_{i=0}^n y_i \cdot \prod_{\begin{smallmatrix}0\le j\le n\\ j\neq i\end{smallmatrix}}
	 \frac{x-x_j}{x_i-x_j}
\]

In this section, I would like to demonstrate how this formula can be obtained
and that it is actually quite straightforward and intuitive.

\subsection{How to arrive at the formula for Lagrange polynomials?}
This formula might seem daunting at first, but let us first consider a
polynomial \( w_i(x) \) that \textbf{goes through only one point \( (x_i, y_i) \) 
and is equal to 0 for the rest of the points}.

% If we find such polynomial for each point in the data set and add them together
% we'll be left with a Lagrange polynomial

\begin{enumerate}
	\item For a given pair \( (x_i, y_i) \), it has to be equal to \( y_i \) 
	\item For each pair \( (x_j, y_j) \) other than \( (x_i, y_i) \), it has to
	be equal to 0
\end{enumerate}

Let's start with condition no. 2. It tells us that the polynomial \( w(x) \)
should be equal to 0 for all \( x_j \) inputs, that are not \( x_i \). That is
actually quite simple -- all we have to do is construct a polynomial in
factored form with the roots being all the \( x_j \) values other than \( x_i \).
\[
	w_i(x) = \prod_{\begin{smallmatrix}0\le j\le n\\ j\neq i\end{smallmatrix}} (x-x_j)
\]

Let's now consider a simplified condition no. 1 -- let's find a polynomial that is
equal to 1 for \( (x_i, x_i) \), but also satisfies condition no. 2. That is
actually also very simple -- we just have to divide the polynomial that we have 
already constructed by \((x_i - x_j)\)
\[
	w_i(x) = \prod_{\begin{smallmatrix}0\le j\le n\\ j\neq i\end{smallmatrix}} \frac{(x - x_j)}{(x_i - x_j)}
\]
Now,
\[
	w_i(x_i) = \prod_{\begin{smallmatrix}0\le j\le n\\ j\neq i\end{smallmatrix}} \frac{(x_i - x_j)}{(x_i - x_j)} = 1
\]
In order for the polynomial to satsify cond. no. 1, all we have to do is 
multiply it by \( y_i \)
\[
	w_i(x) = y_i \prod_{\begin{smallmatrix}0\le j\le n\\ j\neq i\end{smallmatrix}} \frac{(x - x_j)}{(x_i - x_j)}
\]
Condition no. 1 is satisfied,
\[
	w_i(x_i) = y_i \prod_{\begin{smallmatrix}0\le j\le n\\ j\neq i\end{smallmatrix}} \frac{(x_i - x_j)}{(x_i - x_j)} = y_i \cdot 1 = y_i
\]

Since both of the conditions are satisfied, \( w_i(x) \) interpolates one point
-- \( (x_i, y_i) \) and is equal to 0 for all the other points in the dataset. 

Below is a sample dataset comprising three points 
\( (x_1, y_1), (x_2, y_2), (x_3, y_3) \) and the values that polynomials \(w_i\)
take for each of them.

\begin{table}[h!]
    \centering
    \begin{tabular}{c|ccc}
			      & \( x_2 \) & \( x_2 \) & \( x_3 \)\\ \hline
        \( w_1 \) & \( y_2 \) & \(  0  \) & \(  0  \)\\ 
        \( w_2 \) & \(  0  \) & \( y_2 \) & \(  0  \)\\ 
        \( w_3 \) & \(  0  \) & \(  0  \) & \( y_3 \)\\ 
    \end{tabular}
\end{table}

To find a polynomial that interpolates all of the points, we simply create
\( w_i(x) \) for every point and add them together.

\[
	w(x) = \sum_{i = 0}^n w_i(x)
\]

Which expands to	
\[
	w(x) = \sum_{i=0}^n y_i \cdot \prod_{\begin{smallmatrix}0\le j\le n\\ j\neq i\end{smallmatrix}}
	 \frac{x-x_j}{x_i-x_j}
\]

We have arrived at the exact formula for Lagrange polynomials.


\section*{Enclosures} 
\begin{itemize}
	\item File with the program (\texttt{Pietrasik\_Lewandowicz\_Hankus.nb})
\end{itemize}


\end{document}
